%!TeX program = xelatex
\documentclass[12pt,hyperref,a4paper,UTF8]{ctexart}
\usepackage{CUGReport}
\usepackage{listings}
\usepackage{xcolor}
\usepackage{fontspec}
\usepackage{setspace}
\setstretch{1.5} % 设置全局行距为1.5倍

\usepackage{enumitem} % 载入enumitem包以便自定义列表环境
\setlist[itemize]{itemsep=0pt, parsep=0pt} % 设置itemize环境的项目间距和段落间距

\setmainfont{Times New Roman} % 英文正文为Times New Roman

%字号设置
\newcommand{\xiaochuhao}{\fontsize{36pt}{\baselineskip}\selectfont}
\newcommand{\erhao}{\fontsize{21pt}{\baselineskip}\selectfont}
\newcommand{\xiaoerhao}{\fontsize{18pt}{\baselineskip}\selectfont}
\newcommand{\sanhao}{\fontsize{15.75pt}{\baselineskip}\selectfont}
\newcommand{\sihao}{\fontsize{14pt}{18pt}\selectfont}
\newcommand{\xiaosihao}{\fontsize{12pt}{18pt}\selectfont}
%\newcommand{\wuhao}{\fontsize{10.5pt}{18pt}\selectfont}


%封面页设置
{   
    %标题
    \title{ 
        %\vspace{1cm}
        \heiti \xiaochuhao \textbf{{产学研工业协同实践\\实习报告}} \par
        %\vspace{1cm} 
       % \heiti \Large {\underline{XXXXXX进展调研}}    
        \vspace{1cm}
    }

    \author{
        \vspace{0.5cm}
        \kaishu\Large 学生姓名\ \dlmu[9cm]{曾康慧} \qquad \\ %姓名 
        \vspace{0.5cm}
        \kaishu\Large 学\hspace{2em}院\ \dlmu[9cm]{未来技术学院} \qquad \\ %学院
        \vspace{0.5cm}
        \kaishu\Large 班\hspace{2em}级\ \dlmu[9cm]{220211} \\ %班级
        \vspace{0.5cm}
        \kaishu\Large 学\hspace{2em}号\ \dlmu[9cm]{20211003337} \\ %学号
        \vspace{0.5cm}
        %\vspace{0.5cm}
        \kaishu\Large 授课教师\ \dlmu[9cm]{陆承达} \qquad  \\ 
        \vspace{1cm}
      	%\vspace{0.25cm} 
    }
    \date{\today} % 默认为今天的日期,可以注释掉不显示日期
}
%%------------------------document环境开始------------------------%%
\begin{document}

%%-----------------------封面--------------------%%
\cover
\thispagestyle{empty} % 首页不显示页码
%%------------------摘要-------------%%

\newpage
\begin{abstract}
在此填写摘要内容好好好好好好好好好好好好好好好好好好好好好好好好好好好好好好好好好好好好好好好好好好好好好 


\vspace{1cm}
\textbf{关键词:} XXX; \hspace{0.5em} XXX; \hspace{0.5em} XXX
\end{abstract}
\thispagestyle{empty} % 首页不显示页码

%%--------------------------目录页------------------------%%
\newpage
\tableofcontents
 \thispagestyle{empty} % 目录不显示页码

%%------------------------正文页从这里开始-------------------%
\newpage
\setcounter{page}{1} % 让页码从正文开始编号

%%可选择这里也放一个标题
%\begin{center}
%    \title{ \Huge \textbf{{标题}}}
%\end{center}



\section{引言}

\subsection{实习项目的研究背景}

\iffalse
\begin{itemize}
    \item \texttt{main.tex} 主文件
    \item \texttt{reference.bib} 参考文献,使用bibtex
    \item \texttt{ZJUTReport.sty} 文档格式控制,包括一些基础的设置,如页眉、标题、学院、学号、姓名等
    \item \texttt{figures} 放置图片的文件夹
\end{itemize}
\fi

\subsection{实习概况}



\section{实习课题}
\subsection{插入公式}
行内公式$v-\varepsilon+\phi=2$。

插入行间公式如\autoref{Euler}:
\begin{equation}
    v-\varepsilon+\phi=2
    \label{Euler}
\end{equation}

\subsection{插入图片}
莉莉如\autoref{fig:lily}所示,注意这里使用了\verb|~\autoref{}|命令,也就是会自动生成“图”“式”等前缀,无需手动输入。


\begin{figure}[!htbp]
	\centering
	\includegraphics[width=0.7\linewidth]{figures/lily}
	\caption{lily of the valley}
	\label{fig:lily}
\end{figure}

插入上面图片的代码:

\begin{verbatim}
\begin{figure}[!htbp]
	\centering
	\includegraphics[width=0.7\linewidth]{figures/lily}
	\caption{lily of the valley}
	\label{fig:lily}
\end{figure}
\end{verbatim}

\subsection{插入文本框}
本模板定义了一个圆角灰底的文本框,使用简化命令\verb|\tbox{}|即可,如果你不喜欢,可以前往 \texttt{ZJUTReport.sty}对其进行修改。

\tbox{
    这是一个圆角灰底的文本框
}

\subsection{插入表格}
本模板文件如表~\ref{doc} 所示。
\begin{table}[!htbp]
    \centering
    \caption{本模板文件组成}
    \renewcommand{\arraystretch}{1.25} % 增加行间距  
    \begin{tabular}{c@{\hspace{20pt}}c} % 增加列间距  
    
    \hline
        文件名 & 说明 \\
        \hline
        \texttt{main.tex}  & 主文件 \\
        \texttt{reference.bib} & 参考文献 \\
        \texttt{BUAAReport.sty}  & 文档格式控制\\
        \texttt{figures}  & 图片文件夹 \\
        \texttt{code}  & 代码文件夹 \\
        \hline
    \end{tabular}
    \label{doc}
\end{table}



\section{学习与协作、交流情况}

\subsection{学习情况}


\subsection{协作与交流情况}




\section{实习总结}


\subsection{成果及结论}


\subsection{成果及结论}


\subsection{成果及结论}





%\section{定理环境}
%\begin{Theorem}
%\end{Theorem}
%
%\begin{Lemma}
%\end{Lemma}
%
%\begin{Corollary}
%\end{Corollary}
%
%\begin{Proposition}
%\end{Proposition}
%
%\begin{Definition}
%\end{Definition}
%
%\begin{Example}
%\end{Example}
%
%\begin{proof}
%\end{proof}



\subsection{插入参考文献}
直接使用\verb|\cite{}|即可\cite{}。

例如:\cite{}


   \textit{ 此处引用了文献}
   \cite{DBLP:conf/nips/VaswaniSPUJGKP17}。此处引用了文献\cite{DBLP:conf/nips/VaswaniSPUJGKP17}


引用过的文献会自动出现在参考文献中。

\section{写在最后}
\subsection{发布地址}
\begin{itemize}
    \item Github: \url{https://github.com/zjutcvg/ZJUT_Report_LaTeX_Template}
\end{itemize}


%%----------- 参考文献 -------------------%%
%在reference.bib文件中填写参考文献,此处自动生成

\newpage
\bibliographystyle{gbt7714-numerical}  % 引用格式
\bibliography{reference.bib}  % bib源


\newpage
\begin{appendices}

\section{文件列表}

\begin{table}[h]  
	\centering  
	\caption{文件列表}  
	\renewcommand{\arraystretch}{1.25} % 增加行间距  
	\begin{tabular}{c@{\hspace{20pt}}c} % 增加列间距   
		\toprule  
		文件名   & 功能描述 \\
		\midrule  
		q1.m & 问题一程序代码 \\
		q2.m & 问题二程序代码 \\
		q3.m & 问题三程序代码 \\
		q4.m & 问题四程序代码 \\
		\toprule  
	\end{tabular}  
	\label{tab:文件列表}  
\end{table}  
	
	\section{代码}
	\noindent q1.m
	\lstinputlisting[language=matlab]{code/q1.m}
	q2.py
	\lstinputlisting[language=python]{code/q2.py}
	q3.c
	\lstinputlisting[language=c]{code/q3.c}
	q4.cpp
	\lstinputlisting[language=c++]{code/q4.cpp}
\end{appendices}


\end{document}